\chapter{Background}
\section{Switch Mode Power Supply}\label{section-smps}
The root of this approach is based on the electromagnetic interference (EMI) induced by switch mode power supplies (SMPS).\cite{hesener2010electromagnetic,liu2002high} SMPS is a solution proposed by electricians to have more efficient transformers because no current is dissipated as heat by going through resistors.. One of its drawbacks is that it induces EMI on the electrical network and therefore to other appliances, possibly disturbing them. Actually, this drawback can be used to our advantage since each device emits a different EMI which could be seen as its signature. This comes from the fact that there are many types of SMPS, including buck, boost, buck-boost, boost-buck, zeta, charge pump, flyback, RCC, Forward, push-pull and Ćuk. Of course, there are also many different needs in intensity and power for every appliance which make the signature a step further towards its uniqueness.


As an example one of the SMPS, lets take the buck-boost. It is composed of one inductance, one capacitor, one diode, one switch (a transistor), and of course the voltage source and the load. On \autoref{fig:closed_buck-boost} we can see the buck-boost in its closed state. The part in red is the part where current is flowing : the voltage source is loading the inductance, while the capacitor is unloading into the target load. The voltage source can not induce anything in the right part of the schema because of the diode. When the inductance begins to be loaded enough, the voltage change opens the switch.
\begin{figure}[h]
    \centering
    \begin{circuitikz}[scale=2]
    \draw[color=red]
    (0,0) to[voltage source,i^>=$I_s$,v=$V_s$] (0,1)
          to[closing switch] (1,1)
          to[L] (1,0)
          -- (0,0);
    \draw
    (2,1) to[diode] (1,1)
    (2,0) -- (1,0);
    \draw[color=red]
    (2,0) to[C,i^<=$I_c$] (2,1)
          -- (3,1)
          to[generic,label=load,i^<=$I_{load}$,v_<=$V_{load}$] (3,0)
          -- (2,0);
    \end{circuitikz}
    \caption{Closed buck-boost}
    \label{fig:closed_buck-boost}
\end{figure}

We are then in the case shown on \autoref{fig:open_buck-boost}. Again, the part where the current flows is in red. Now, the inductance is unloading into the capacitor and the target load. The capacitor will then be able to be to unload again when the switch will be open.


\begin{figure}[h]
    \centering
    \begin{circuitikz}[scale=2]
    \draw
    (0,0) to[voltage source,i^>=$I_s$,v=$V_s$] (0,1)
          to[opening switch] (1,1)
    (1,0) -- (0,0);
    \draw[color=red]
    (1,1) to[L,i^>=$I_l$] (1,0)
    (2,1) to[diode] (1,1)
    (2,0) -- (1,0);
    \draw[color=red]
    (2,0) to[C,i_>=$I_l$] (2,1)
          -- (3,1)
          to[generic,label=load,i^<=$I_{load}$,v_<=$V_{load}$] (3,0)
          -- (2,0);
    \end{circuitikz}
    \caption{Open buck-boost}
    \label{fig:open_buck-boost}
\end{figure} %TODO sens des fleches ?

The percentage of time that the switch is open is called the duty cycle and controls the potential difference around the load. Because the power of the load may vary, the uptime and downtime are not the same for a fixed duty cycle and source voltage.

Every time the switch is opened or close, there are repercussions on the electrical network.

Also, because of regulations, appliances are equipped with EMI filters. Although they reduce the higher peaks, they also induce smaller variations in the EMI spectrum that make the EMI more unique for every appliance. This is again due to a lot of different filter types and their interractions with the SMPS.



\subsection{Electromagnetic Interference}
Switch mode power supply can induce electromagnetic interferences (EMI) by multiple means. The resistors affect the whole range of frequency, but capacitors and inductances mainly affect the higher frequencies, up to a few Ghz.

Because not all the parts of appliances need the same input voltage, there could be multiple SMPS inside an appliance. But even with only one SMPS, the duty cycle of the converter coupled with the type of converter and the power of the appliance unload very different EMI on the network.



Also, no capacitor is ideal, there are equivalent series resistance (ESR) and inductance (ESL) used to represent it. When current goes trough the ESR, there is a voltage drop delayed by the inductance. %TODO is that true ? 
This has an effect on the network as well.
\begin{figure}[h]
    \centering
    \begin{circuitikz} \draw
    (0,0) to[R,label=ESR,o-] (2,0) to[L,label=ESL] (4,0) to[C,-o] (6,0);
    \end{circuitikz}
    \caption{Circuit model of non ideal capacitor}
    \label{fig:non_ideal_capacitor}
\end{figure}





\section{Fourier Transform}
The Fourier transform 