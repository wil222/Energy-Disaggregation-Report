\chapter{Conclusion}
Throughout this thesis, we have seen that there are multiple approaches to disaggregate the energy consumption. The first methods that have been tried, based only on the instantaneous power consumption readings weren't as promising as the more recent methods using high frequency (range of kHz) and very high frequency (range of MHz) sampling of the current intensity.

The intention of the thesis was to reproduce an algorithm that allowed people to easily set up their own system, by making a solution easy to set up. I wanted also it to be as cheap as possible. I think that both these goals are fulfilled. Actually all the requirements defined in \autoref{section:requirements} are met. The hardware is easy to get, the code is available\footnote{\url{http://gitlab.com/william.andre/EnergyDisaggregation}} and written in a well-known language (Python). It is easy to communicate the information about the disaggregation to another software, the \acrshort{cpu} usage is low and the response time is less than a few seconds or even faster.

Even though this thesis was greatly inspired by ElectriSense \cite{gupta2010electrisense} and Smappee \cite{bruneel2018energy}, I have searched for improvements. Obviously, the main difference with ElectriSense is that they are using samplings on a much larger frequency band. That larger band allows to have a greater precision, but it also means higher costs which can be an obstacle.\\
The differences with Smappee are both the event detection algorithm and the event association algorithm. While the comparison of those algorithms is difficult because of the lack of automatic learning and automatic testing, the event detection algorithm has at least an advantage regarding its time complexity. This could allow putting more sensors, as suggested in \autoref{chapter-improvements}, because the algorithm running the more often (besides the \acrshort{fft} that is used in both projects) is that event detection; the event association only runs when an event is detected. Also, the method in this thesis can detect multiple different states for one appliance when Smappee only detects on/off states.

However, there are still limitations to this technique. These are mainly the lack of an automated learning and the imprecision of the sensor because of its maximum range. I hope and I believe that this will be improved in the future.